\documentclass[../main]{subfiles}
\begin{document}

\chapter{Algebra, Number Theory and Related Assumptions}

\begin{lemma}
		If $n, m \in{} \mathbb{N}$ and $m > 1$, then n is invertible modulo m (there exists p such that np mod m = 1) whenever gcd(n, m) = 1, i. e. when n, m are comprime.
\end{lemma}

\noindent
$\textbf{wFactor}_A (n):$\\
$(x, y) \leftarrow{} \mathbb{N} \times{} \mathbb{N} \; with \; |x| = |y| = n$\\
$N \leftarrow{} x \cdot{} y$\\
$(z, w) \leftarrow{} A(N)$\\
$\textbf{Result:} \; z \cdot{} w = N$\\
$$Pr(wFactor_A (n) = 1) = \varepsilon{} (n)$$

\noindent
$\textbf{for} \; i \leftarrow{} 1 \; to \; t \; \textbf{do}$\\
$\quad{}\quad{}r \leftarrow{} \{0, 1\}^{n-1}$\\
$\quad{}\quad{}p \leftarrow{} 1\mid\mid r$\\
$\quad{}\quad{}\textbf{if} \; p \; is \; prime \; \textbf{then}$\\
$\quad{}\quad{}\quad{}\quad{}\textbf{Result:} \; p$\\
$\textbf{Result:} \; fail$\\

\begin{theorem}
	There exists a constant c such that for every $n > 1$ the number of primes that can be represented in exactly n bits is at least equal to $\frac{c \cdot{} 2^{n-1}}{n}$.
\end{theorem}

\begin{theorem}
	If $(\mathbb{G}, \cdot)$ has order m, then for each $g \in{} \mathbb{G}$, it is true that $g^m = 1_{\mathbb{G}}$.
\end{theorem}

\begin{corollary}
	If $(\mathbb{G}, \cdot)$ has order $m > 1$, then for every $g \in{} \mathbb{G}$ and for every i, $g^i = g^{[i \; mod \; m]}$.
\end{corollary}

\begin{theorem}
	Let $N > 1$. For every natural $e > 0$, we define $f_e : \mathbb{Z}_N* \rightarrow{} \mathbb{Z}_N*$ assuming $f_e (x) = x^e \; mod \; N$.
	If $gcd(e, \Phi (N)) = 1$, then $f_e$ is a permutation. Moreover, if d is the inverse of e (modulo $\Phi (N)$), then $f_d$ is the inverse of $f_e$.
\end{theorem}

\noindent
$\textbf{RSAInv}_{A, GenRSA}(n):$\\
$(N, e, d) \leftarrow{} GenRSA(1^n)$\\
$y \leftarrow{} \mathbb{Z}_N*$\\
$x \leftarrow{} A(N, e, y)$\\
$\textbf{Result:} \; x^e mod N = y$
$$Pr(RSAInv_{A, GenRSA}(n) = 1) = \varepsilon{} (n)$$

\noindent
$(N, p, q) \leftarrow{} \textbf{GenModulus}(1^n)$\\
$M \leftarrow{} (p-1)(q-1)$\\
$e \leftarrow{} \{1, \ldots, M\} \; such \; that \; gcd(e, M) = 1$\\
$d \leftarrow{} e^{-1} mod M$\\
$\textbf{Result:} \; (N, e, d)$

\begin{lemma}
	If $\mathbb{G}$ has order m and $g \in{} \mathbb{G}$ has order i, then $i|m$.
\end{lemma}

\begin{theorem}
	If $\mathbb{G}$ has prime order then $\mathbb{G}$ is cyclic and every $g \in{} \mathbb{G}$ with $g \neq{} 1_{\mathbb{G}}$ generates $\mathbb{G}.$
\end{theorem}

\noindent
$\textbf{DLog}_{A, GenCG} (n):$\\
$(\mathbb{G}, q, g) \leftarrow{} GenCG(1^n)$\\
$h \leftarrow{} \mathbb{G}$\\
$x \leftarrow{} A(\mathbb{G}, q, g, h)$\\
$\textbf{Result:} \; g^x  =    n $
$$Pr(DLog_{A, GenCG} (n) = 1) = \varepsilon{} (n)$$

% slides 43-49

\begin{theorem}
	If factoring is hard relative to \textbf{GenModulus}, then $f_{GenModulus}$ is a one-way function.
\end{theorem}

% Proof 16/11/2021
\begin{theorem}
	If $\mathbb{G}$ is a finite group where $m = |\mathbb{G}|$ is the order of $\mathbb{G}$, then for every $g \in{} \mathbb{G}$, it holds that $g^n = 1_{\mathbb{G}}$.
\end{theorem}

\paragraph{Proof}
	\begin{enumerate}
		\item Let us make the assumption that $\mathbb{G}$ is abelian (this of course makes the proof less general, but has the advantage of simplifying it).
		\item Take $\mathbb{G} = \{g_1, \ldots, g_m\}$ and consider any $g \in{} \mathbb{G}$.
		\item Of course
				$$g_1 \cdot{} g_2 \cdot{} g_3 \cdot{} \ldots{} \cdot{} g_n = (gg_1)\cdot{}(gg_2)\ldots(gg_m)$$
				Indeed, the factors in the RHS are all distinct elements of $\mathbb{G}$ because if $gg_i = gg_j$; then $g^{-1}gg_i = g^{-1}gg_j$ and $g_i = g_j$, which would contradict the fact that $|\{g_1, \ldots, g_m\}| = m$.
		\item Since $\mathbb{G}$ is abelian, we can rearrange the rhs of \textcolor{red}{*} and obtain that
				$$g_1g_2 \cdots{} g_m = g^m \cdot{} (g_1 \cdot{} g_2 \cdots g_m)^{-1}$$
		\item We can then multiply both sides of this equation by $(g_1 g_2 \cdots{} g_m)^{-1}$ and obtain:
				\begin{align*}
					1_{\mathbb{G}} &= (g_1 g_2 \cdots{} g_m) \cdot (g_1 g_2 \cdots{} g_m)^{-1}
					% Complete with recorder
				\end{align*}
	\end{enumerate}

\end{document}	