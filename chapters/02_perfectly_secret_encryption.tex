\documentclass[../main]{subfiles}
\begin{document}

\chapter{Perfectly secret encryption}

\section{Perfect secrecy}

\begin{definition}[Perfect Secrecy]
    An encryption scheme (\textit{Gen, Enc, Dec}) is perfectly secret if for every message $m \in{} \mathcal{M}$ and every ciphertext $c \in{} \mathcal{C}$ for which $Pr(C=c)>0$ we have that $$Pr(M=m \mid{} C=c) = Pr(M=m)$$
\end{definition}

\begin{lemma}[Flips previous equation]
    \label{lemma:perfect-secrecy-first-characterization}
    An encryption scheme is perfectly secure if and only if for every distribution on $m \in{} \mathcal{M}$ and for every ciphertext $c \in{} \mathcal{C}$ such that $Pr(C=c) > 0$, we have that $$Pr(C=c \mid{} M=m) = Pr(C=c)$$
\end{lemma}

\paragraph{Proof}
    \begin{itemize}
        \item[$\Leftarrow{}$]
            \begin{align*}
                \cancel{Pr(C=c)} &= Pr(C=c \mid M=m) \\
                &= \frac{\cancel{Pr(C=c)} \cdot Pr(M=m \mid C=c)}{Pr(M=m)} &&\text{(Bayes theorem)}
            \end{align*}
            As a consequence
            \begin{align*}
                Pr(M=m) &= Pr(M=m \mid C=c)
            \end{align*}
        \item[$\Rightarrow{}$]
            The proof is identical, except that the role of the event $C=c$ is played by $M=m$, and viceversa.
    \end{itemize}
    We have therefore proved \ref{lemma:perfect-secrecy-first-characterization}.
%----------------------------------------------------------------------------------
\begin{lemma}[Same probability for every pair of messages]
    \label{lemma:perfect-secrecy-second-characterization}
    An encryption scheme is perfectly secure if for every distribution on $\mathcal{M}$ and for every pair of messages $m_0, m_1 \in{} \mathcal{M}$ and for every ciphertext $c \in{} \mathcal{C}$, it holds that $$Pr(C=C \mid{} M=m_0) = Pr(C=c \mid{} M = m_1)$$
\end{lemma}

\newpage

\paragraph{Proof}
    \begin{enumerate}
        \item[$\Rightarrow{}$]
        This implication is easy to be proved:
            \begin{align*}
                Pr(C=c \mid M=m_0) &= Pr(C=c) \\
                &= Pr(C=c \mid M=m_1)
            \end{align*}
        We are applying the previous lemma because the underlying encryption scheme is perfectly secure.
        \item[$\Leftarrow{}$]
        We can assume that given $c \in \mathcal{C}$, every term in the form
        \[
            Pr(C=c \mid M=m_i)
        \]
        have the same value. Call this value $p_c \in \mathbb{R}_{[0, 1]}$ we can then write that
            \begin{align*}
                Pr(C=c) &= \sum_{m_i \in M} Pr(C=c \mid M=m_i) \\
                &= \sum_{m_i \in M} Pr(C=c \mid M=m_i) \cdot Pr(M=m_i) && \text{(equal by prob theory)} \\
                &= \sum_{m_i \in M} p_c \cdot Pr(M=m_i) \\
                &= p_c \cdot \sum_{m_i \in M} Pr(M=m_i) && \text{(sum is equal to 1)} \\
                &= p_c \cdot 1 \\
                &= p_c \\
                &= Pr(C=c \mid M=m)
            \end{align*}
    \end{enumerate}

    We have therefore proved \ref{lemma:perfect-secrecy-second-characterization}.
%----------------------------------------------------------------------------------
\section{One-Time Pad (aka Vernam's Cipher)}
\begin{definition}[OTP]
    $\mathcal{M} = \mathcal{C} = \mathcal{K} = \{0, 1\}^n$ where
    \begin{align*}
        & Gen \leftarrow{} \{0, 1\}^n \\
        & Enc(k,m) = m \oplus k \\
        & Dec(k,c) = c \oplus k \\
        & \text{Gen returns 000, 001, $\cdots$, 111} && \text{(n=3, each with prob $\frac{1}{8}=\frac{1}{2^n}$)} \\
        & Enc(100, 001) = 101 \\
        & Dec(110, 011) = 101
    \end{align*}
\end{definition}

\begin{theorem}[OTP perfectly-secret]
    \label{theorem:otp-perfectly-secure}
    The OTP is perfectly secure (perfectly-secret).
\end{theorem}

\paragraph{Proof}
    Let's fix a distribution on $\mathcal{M}$ and let's also fix a message $m \in{} \mathcal{M}$ and a ciphertext $c \in{} \mathcal{C}$.
    Consider now the probability of $C=c$ given that $M=m$: we would like to prove that this does \underline{not} depend on $m$.
    \begin{align*}
        Pr(C=c \mid{} M=m) &= Pr(M \oplus{} K=c \mid{} M=m) && \text{(substitute C as rand var)} \\
        &= Pr(m \oplus{} K=c) && \text{(M and K are independent)} \\
        &= Pr(m \oplus{} (m \oplus{} K) = m \oplus{} c) && \text{($m \oplus{} x = m \oplus{} y$ if $x=y$)} \\
        &= Pr((m \oplus{} m)\oplus K = m \oplus{} c) && \text{($m \oplus m$ is $0^n=1$)} \\
        &= Pr(K=m \oplus{} c) \\
        &= \frac{1}{2^n}
    \end{align*}
    We are happy now because we can write $$Pr(C=c \mid{} M=m_0) = \frac{1}{2^n} = Pr(C=c \mid{} M=m_i)$$
    As a consequence, the OTP is perfectly secure and therefore \ref{theorem:otp-perfectly-secure} is proved.
%----------------------------------------------------------------------------------
\section{Shannon's Theorem}

\begin{theorem}[Key set is greater or equal of message set]
    \label{theorem:cardinality-key-message-set}
    If an encryption scheme is perfectly secure, then $\mid{} \mathcal{K}\mid{} \geq \mid{} \mathcal{M}\mid{} $.
\end{theorem}

\paragraph{Proof}
    \begin{itemize}
        \item By way of contradiction. Let us assume that some concrete perfectly secure encryption scheme exists such that $\mid{} \mathcal{K}\mid{} < \mid{}\mathcal{M}\mid{}$, call it $\Pi$
        \item From the fact that $\Pi$ is perfectly secure, we know that it must satisfy the definition of perfect security for every distribution on $\mathcal{M}$. So, suppose that every $m \in{} \mathcal{M}$ is equally likely to occur. Let's also pick one ciphertext $c \in{} \mathcal{C}$ such that $Pr(C=c) > 0$
        \item Let us define $\mathcal{M}(c) \hat{=} \{\hat{m} \mid{} \hat{m} = Dec(k,c)$ $\forall k \in{} \mathcal{K}$\} and let's evaluate $\mid{} \mathcal{M}(c) \mid{}$:
        \begin{itemize}
            \item[$\circ$] $\mid{}\mathcal{M}(c)\mid{} \leq \mid{}\mathcal{K}\mid{} < \mid{}\mathcal{M}\mid{}$
        \end{itemize}
        \item In other words, there must be a message $m' \in{} \mathcal{M}$ such that $m' \notin{} \mathcal{M}(c)$, $$Pr(M=m' \mid C=c) = 0 \text{(because it's impossible)} < Pr(M=m')$$
    \end{itemize}
    Since $\Pi$ was supposed to be perfectly secure, we have reached a contradiction and therefore \ref{theorem:cardinality-key-message-set} is proved.
%----------------------------------------------------------------------------------
\newpage
\section{Indistinguishability}

This characterization doesn't rely on probability theory, it's more computational.
We can consider a program called $PrivK^{eav}_{A, \Pi}$ where:
\begin{itemize}
    \item the abbreviation $eav$ stands for eavesdropper (ciphertext-attack);
    \item $A$ is the adversary that tries to force the scheme;
    \item the scheme $\Pi$ is defined as usual with the triple $(Gen, Enc, Dec)$.
\end{itemize}
In this game the adversary $A$ ask for a pair of messages, it generates a key calling $Gen$ and then generate a random bit called $b$ that takes value 0 or 1 with probability $\frac{1}{2}$ each. After, we encrypt the message $m_b$ according to $k$ key and we ask to adversary to guess which one of the two messages was encrypted. The result is 1 if the adversary correctly guess the value of $b$ or 0 otherwise.\\
$PrivK^{eav}_{A,\Pi}$:\\
$(m_0,m_1) \; \leftarrow{} \; A$\\
$k \; \leftarrow{} \; Gen$\\
$b \; \leftarrow{} \; \{0,1\}$\\
$c \; \leftarrow{} \; Enc(k,m_b)$\\
$b^* \; \leftarrow{} \; A(c)$\\
\textbf{Result}: $\neg(b \oplus{} b^*)$

\begin{definition}[Indistinguishable encryption]
    An encryption scheme $\Pi = (Gen, Enc, Dec)$ has indistinguishable encryptions if and only if for every adversary $A$ we have that:$$Pr(PrivK^{eav}_{A,\Pi} = 1) = \frac{1}{2}$$
\end{definition}

\begin{theorem}[Strong characterization]
    $\Pi$ is perfectly secret if and only if $\Pi$ has indistinguishable encryptions.
\end{theorem}

\end{document}